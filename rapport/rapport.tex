\documentclass[12pt]{article}
\usepackage[utf8]{inputenc}
\usepackage[T1]{fontenc}
\usepackage[french]{babel}
\usepackage{amssymb,amsbsy,amsmath,amsfonts,amscd}
\usepackage{geometry}
 \geometry{
 a4paper,
 total={170mm,257mm},
 left=20mm,
 top=0mm,
 }
\usepackage{color}
\usepackage{hyperref}
\usepackage{graphicx}
\usepackage{pgfpages}
\usepackage{blindtext}
\pgfpagesuselayout{2 on 1}[a4paper]


\begin{document}


\title{\includegraphics[width=4cm]{LOGO_SCIENCES_DEF_CMJN.jpg}\\Rapport projet ADD sur jeu de donn\'ees Rossignol}
\author{Mohamed SOUANE\\
        Damian BIMBENET\\
        Bailin CAI\\
        Alexandre ZAFARI\\
}
\date{\today}
\maketitle

\section{Introduction}

\'Etude d'une population de cette espèce d'oiseau en Suède. Pour prédire, la réponse
évolutive de la sélection dans ces traits, on présente des estimations e corrélations d'héritabilité $h^2$

\section{Analyse Descriptive}
\begin{itemize}
\item on commence par introduire les donn\'ees
\end{itemize}

\subsection{Pr\'e-traitement des données}
% dans \item ajouter les phrases
\begin{itemize}
    \item on transf\'erer les donn\'ees quatitatives sous formes numériques
\end{itemize}


\subsection{Description des données}
% ajouter les description des données
\begin{itemize}
    \item c'est la bases des donn\'ees sur les oiseaux
    \item S\'eparer les donn\'ees en 2 groupes male et femelle
    \item Tracer les histogrammes conditionnelle
    \item Faire les tests de normalit\'e
    \item Afficher les boxplot s\'eparer en 2 groupes
    \item Faire les tests de comparaison de la variance
    \item Regarder la corr\'elation entre les variables quantitatives
    \item Regarder la corr\'elation en 2 groupes male et femelle
    \item Test de corr\'elation dans les 2 groupes
    \item Analyse PCA
\end{itemize}



\section{La regression lin\'eaire}
\begin{itemize}
\item \`A partir de la matrice de la corr\'elation, on fait d'abord la regression lin\'eaire simple
\item Puis la regression multiples
\end{itemize}


\section{Probl\`eme classification} 
\begin{itemize}
\item 
\end{itemize}

\section{Conclusion} 


  
% Prints all the non-cited references

% Use style 'alphakey' or 'alpha' for the draft, and then switch 
% to 'unsrt' or 'plain' or 'ieeetr' styles for the final version, 
% since they are the IEEE preferred ones. 

%\bibliographystyle{alpha}



\end{document}
