\documentclass[12pt, twocolumn]{article}
\usepackage[utf8]{inputenc}
\usepackage[T1]{fontenc}
\usepackage[french]{babel}
\usepackage{amssymb,amsbsy,amsmath,amsfonts,amscd}
\usepackage{geometry}
\usepackage{tabularx}
\usepackage{adjustbox}
\usepackage{tabulary}
 \geometry{
 a4paper,
 total={170mm,257mm},
 left=20mm,
 top=0mm,
 }
\usepackage{color}
\usepackage{hyperref}
\usepackage{graphicx}
\usepackage{pgfpages}
\usepackage{blindtext}


\begin{document}


\title{\includegraphics[width=4cm]{LOGO_SCIENCES_DEF_CMJN.jpg}\\Rapport projet ADD sur jeu de donn\'ees Rossignol}
\author{Mohamed SOUANE\\
        Damian BIMBENET\\
        Bailin CAI\\
        Alexandre ZAFARI\\
}
\date{\today}
\maketitle

\section{Introduction}

\'Etude d'une population de cette espèce d'oiseau en Suède. Pour prédire, la réponse
évolutive de la sélection dans ces traits, on présente des estimations e corrélations d'héritabilité $h^2$

\section{Analyse Descriptive}

L'âge moyen des $544$ sujets inclus dans la cohorte était de 32 mois ($q_1 = 20; q_3 = 35$), $35\%$ étaient des femelles et $65\%$ étaient des mâles.
Longueur des ailes moyen mesuré à l'inclusion était de 99cm. Le poids moyen mesuré était $33,9$g. La longueur de la queue moyen mesuré était de 33cm. La hauteur du bec
moyen mesuré était de 5cm. La largeur du bec moyen mesuré étatit de 5cm.


\subsection{Visualisation des données}
% dans \item ajouter les phrases
Décrivez et comparez à l'aide de tests statistiques les caractéristiques dans les deux groupes mâle et femelle. L'objectif ici est de décrire et de comparer les groupes
mâle et femme en fonction des caractéristiques recuiillies qui sont toutes des variables quantitatives, en utilisant un test de comparaison de deux moyennes. les effectifs dans 
chaque groupe étant suffisant grande. Le test de wilcoxon est assez robuste au non-respect de la normalité des distributions.

Les box-plot montre qu'on ne peut pas confirmer que le sexe a une influence sur l'âge, la projection des ailes, la hauteur de la bec, la largeur de la bec, la longueur de la bec, le poids.
Par contre il semble avoir une sur la longueur de la queue, la longueur du tarsus, la longueur des ailes.

\begin{table}[h!]
    \centering
    \begin{tabular}{||c c c c||} 
     \hline
     Variable & Mâle(mean) & Femelle(mean) & P-value \\ [0.5ex] 
     \hline\hline
     Poids & 6 & 87837 & 787 \\ 
     2 & 7 & 78 & 5415 \\
     3 & 545 & 778 & 7507 \\
     4 & 545 & 18744 & 7560 \\
     5 & 88 & 788 & 6344 \\ [0.5ex] 
     \hline
    \end{tabular}
    \caption{Test comparaisons des caractéristiques chez les mâles et femelles}
    \label{table:1}
\end{table}


\paragraph{Commentaires Tableau 1}
Comme on pouvait s'y attendre, le poids en moyenne est significativement plus basses chez le sfemelles que chez les mâles (p-value $< 0,5$).
Il en ait de même pour la longueur des ailes, la longueur des tarsus, la longueur de la queue, la longueur de la bec, la hauteur de la bec, la projection des ailes.
Par contre, les deux groupes ne diffèrent pas significativement pour l'âge. La largeur des becs en moyenne est significativement plus basses chez les mâles que chez les femelles (p-value $> 0,5$)

\newpage

\subsection{Matrice de corrélation}

\begin{table}[h!]
    \centering
    \begin{adjustbox}{width=0.4\textwidth}
    \small
    \begin{tabular}{||c c c c c||} 
     \hline
     Variable & Mâle(coeff) & P-value & Femelle(coeff) & P-value \\ [0.5ex] 
     \hline\hline
     Poids & 6 & 87837 & 787 & 1\\ 
     2 & 7 & 78 & 5415 & 1\\
     3 & 545 & 778 & 7507 & 1\\
     4 & 545 & 18744 & 7560 & 1\\
     5 & 88 & 788 & 6344 & 1\\ [0.5ex] 
     \hline
    \end{tabular}
\end{adjustbox}
    \caption{Le test du coefficient de corrélation linéaire entre la longueur du tarsus et certaines traits morphologiques chez les mâles et chez les femelles}
    \label{table:2}
\end{table}

\paragraph{Commentaires Tableau 2}

D'après les résultats du Tableau \ref{table:2} et les deux matrice de corrélation, chez les mâles, seules les variables longeueur des ailes, longueur du bec, longueur de la queue, hauteur de la bec, largeur
de la bec, poids, sont significativement liées linéairement à la PAM($ p < 0,5 $). Les liens sont positifs. On retrouve les mêmes résultats chez les femelles en ce qui concerne longueur des ailes, longueur du bec, largueur de la bec et poids,
mais pas pour la longueur des ailes, longueur du vbec, largueur de la bec ($p = 0,097$). L'âge et la projection des ailes ne sont pas significativement liées linéairement à la longueur du tarsus dans les deux groupes.

\section{La regression linéaire multiple}
Dans cette partie on a réalisé le modèle avec toutes les variables quantitatives, 

\subsection{Le choix des variables}
Pour le choix du modèle, nous vous appliqué une procédure de simplification descendante pas à pas. La procédure de la simplification
descendante pas à pas est une approche visant à améliorer le modèle explicatif. On réalise un premier modèle avec toutes les variables spécifiées,
puis on regarde s'il est possible d'améliorer le modèle avec toutes les variables spécifiées, puis on regarde s'il est possible d'améliorer le modèle
en supprimant une des variales dont la suppression améliorera les plus le modèle. Puis on recommence le même en supprimant une des variables dont la suppression améliorera le plus le modèle.
Puis on recommence le même procédure pour voir si la suppression d'une seconde variables peut encore améliorer le modèle eet ainsi de suite. Lorsque le modèle ne peut plus être amélioré par la suppression
d'une variable on s'arrête.

\begin{table}[h!]
    \centering
    \begin{tabular}{||c c c c||} 
     \hline
     Variable & Mâle(coeff) & P-value & Femelle(coeff) \\ [0.5ex] 
     \hline\hline
     Poids & 6 & 87837 & 787 \\ 
     2 & 7 & 78 & 5415 \\
     3 & 545 & 778 & 7507 \\
     4 & 545 & 18744 & 7560 \\
     5 & 88 & 788 & 6344 \\ [0.5ex] 
     \hline
    \end{tabular}
    \caption{Relation entre le poids et certaines traits morphologiques. Analyse multivariable. Régression linéaire}
    \label{table:3}
\end{table}

\subsection{Analyse des résidus}
Cette étape concerne la vérification des hypothèses énoncées sur l'erreur estimée. Il s'agit du test de normalité du résidu. La validation des modèles est attachée à
l'idée selon laquelle les résidus sont indépendants et identiquement distribuées suivant la loi normale centrée avec un variance constante.

% Le figure

L'alignement des points sur la première bisserctrice est presque parfaite, on en déduit l'hypothèse selon laquelle les résidus théoriques suivent une loinormale mais pas centrée réduite.

% Le figure

D'après les graphiques ci-dessus, $3.6\%$ des résidus sont aberrants, ce qui est acceptable pour un jeu de données de taille $544$.
De plus, ces résidus ne sont pas très éloignés des limites d'intervalle. Ce ne sont pas de véritables observations oberrantes, car $96,4\%$ des
observations sont entre les deux seuils. $6\%$ sont des points leviers et un seul distance de Cook élévée(observation 26 est à la fois levier et aberrante).



\section{ANOVA} 


\section{Conclusion} 


  
% Prints all the non-cited references

% Use style 'alphakey' or 'alpha' for the draft, and then switch 
% to 'unsrt' or 'plain' or 'ieeetr' styles for the final version, 
% since they are the IEEE preferred ones. 

%\bibliographystyle{alpha}



\end{document}
